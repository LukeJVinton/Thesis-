

This Chapter discusses three methods that may improve the
sensitivity of the muon neutrino disappearance analysis.
The aim of each method is to decrease the size of the sensitivity
contour ($\Delta m^2$ vs. $\sin^2\theta_{23}$), if the size of the
sensitivity contour is reduced then the oscillation parameters can
typically be measured with greater precision.
The sensitivity contour is obtained from fitting
a simulated far detector spectrum to an oscillated fake data
spectrum. 

The first method involves separating neutrino events into bins of
energy resolution such that well resolved events are not polluted by
the less well resolved sample. 
In the second method the neutrino energy binning is altered to provide
finer binning in the region of interest.
The third method utilises a hybrid of the ReMId
(as used in NOvA's first and second muon neutrino disappearance
analyses) and CVN selectors.
This chapter consists of three main discussions.
First, the individual optimisation of the above methods. 
Next, the combination of the individually optimised methods. 
Finally, the combined optimisation of the methods.


\section{Best Fit Points}

The sensitivity of the experiment is tested with
altered oscillations applied to the fake data spectrum to sample the
sensitivity at a range of potential neutrino oscillation parameters. 

The impact of each method will be measured using three metrics: the
sensitivity to reject maximal mixing ($\sin^2\theta_{23} = 0.5$) for
the NOvA second analysis 
$\nu_\mu$ disappearance best fit, the sensitivity to reject maximal
mixing for the most recent MINOS $\nu_\mu$ disappearance best fit and
finally the sensitivity to reject $\sin^22\theta_{23} = 0.6$ for the most
recent T2K $\nu_\mu$ disappearance best fit. All Best fit points are
taken assuming the Normal Ordering hypothesis.

The MINOS best fit is taken from the 2014 combined analysis
paper~\cite{adamson2014combined}. 
This paper reported measurements of 
$|\Delta m^2_{32}| = [2.28 - 2.46] \times 10^{-3}~\text{eV}^2~(68\%
C.L.)$ and $\sin^2\theta_{23} = 0.35–0.65 \text{(90\% C.L.)}$. 
In the following analyses the best fit point used to make
the sensitivity contours for comparisons is taken to be $|\Delta
m^2_{32}| = 2.36 \times10^{-3}~\text{eV}^2$ and $\sin^2\theta_{23} =
0.41$.  
The T2K best fit point is extracted from the results of the 2015
paper~\cite{abe2015measurements}. The reported measurements are
$\sin^2\theta_{23}=0.528^{+0.055}_{-0.038}$ and $|\Delta
m^2_{32}|=(2.51 \pm 0.11)\times10^{-3}~\text{eV}^2)$
NOvA's second analysis best fit point was presented in the 2017
disappearance paper~\cite{NOvASA}. The normal ordering, lower octant
best fit point is $\sin^2\theta_{23} = 0.404^{+0.030}_{-0.022}$ and
$|\Delta m^2_{32}|=(2.67 \pm 0.11)\times10^{-3}~\text{eV}^2)$. 



\section{Impact of Cosmic Background on Sensitivity Contours}

Figure~\ref{fig:stdContourWithCosmics} shows the sensitivity of the
standard disappearance analysis to the mixing parameters $\Delta
m^2_{32}$ and $\sin^2\theta_{23}$. The sensitivity when accounting for
the cosmic background (blue contour) is compared with the sensitivity
neglecting the cosmic background (red contour). 
The contours show that the addition of the cosmic background
significantly reduces the sensitivity of the experiment to rejecting
maximal mixing.  

The cosmic background will not initially be included for the following
studies designed to optimise the sensitivity of the analysis. 
Instead, the impact of the cosmic background on the sensitivity with
all the proposed analysis improvements included will be presented
later in the chapter.



\begin{figure}
  \centering
\includegraphics[width=0.9\textwidth]{../../../Analysis/thesisPlots/anaImprovs/stdPlots/contour_compCosmics.pdf}
  \caption{
    Comparison of sensitivity with (blue contour) and without (red
    contour) accounting for the cosmic background at NOvA's SA best fit
    point.
  } 
  \label{fig:stdContourWithCosmics}
\end{figure}



\section{Hadronic Energy Fraction Binning}

The first sensitivity improvement considered is to separate events by
reconstructed neutrino energy resolution. 
The reconstructed neutrino energy is the sum of the reconstructed
muon energy and the reconstructed hadronic shower energy.
In the NOvA detectors, muon energy is estimated using the length of
the muon track with a resolution of 2\% while hadronic energy is
estimated from the calorimetric response of the detector to a cluster
of cell-hits with a resolution of about 25\%~\cite{NOvASA}. 
Therefore, neutrino events with a larger proportion of hadronic
activity will be less well resolved by the detector.
Events with better energy resolution are less likely to migrate
across neutrino energy bin boundaries. This is particularly important
to the result of the analysis for events near the oscillation dip,
events migrating into or out of the oscillation dip will alter the
final measurement. 


In the following Section hadronic energy fraction, $E_{had.} /
E_{\nu}$, is used as a metric to estimate the neutrino energy
resolution. 
The $E_{had.} / E_{\nu}$ distribution is obtained for each bin of
reconstructed neutrino energy and the quantiles of $E_{had.}/E_{\nu}$
are identified.  
These quantiles are then used as the boundaries of the $E_{had.} /
E_\nu$ bins. 
In this analysis the number of \hadefrac{} bins can be varied to
optimise the sensitivity.  
Figures~\ref{fig:hadEFracHistP1}, \ref{fig:hadEFracHistP2},
\ref{fig:hadEFracHistE3b} 
and \ref{fig:hadEFracHistE3c} show the distribution of
hadronic energy fraction vs. reconstructed neutrino energy for each
far detector running period used in this analysis. 
The quantile boundaries are formed on a period by period basis to
incorporate any changes in the detector performance such as the
introduction of a higher APD gain in the far detector since the 2015
summer shutdown. The far detector was set to a higher gain since the
start of running period 3.
To provide an example, the quantile
boundaries for a division into four hadronic energy fraction quantiles
are shown by the black histograms overlaid on the colour plot. 
In this case, with four \hadefrac{} bins, the lowest \hadefrac{}
quantile contains the
25\% of the sample with the best neutrino energy resolution. On the
other hand, the highest quantile contains the 25\% of the sample with
the worst neutrino energy resolution.


The sensitivity contours for scenarios where the far detector neutrino
energy spectrum is split into 2, 3, 4, 5, 6, 7 and 8 quantiles of
\hadefrac{} are compared with the standard analysis
at NOvA's second analysis best fit point in
Figure~\ref{fig:hadEFracContour_NOvA}, at MINOS's best fit point in
Figure~\ref{fig:hadEFracContour_MINOS} and at T2K's latest best fit
point in Figure~\ref{fig:hadEFracContour_T2K}. In addition, the
significance of rejection of a point in oscillation parameter space
(maximal mixing for the NOvA and MINOS best fit and $\sin^2\theta_{23}
= 0.6$ for the T2K best fit) is plotted against 
the number of \hadefrac{} quantiles for the NOvA, MINOS and T2K best fit
points in
Figures~\ref{fig:hadEFracRej_NOvA},~\ref{fig:hadEFracRej_MINOS} and
~\ref{fig:hadEFracRej_T2K} respectively.

% Discuss the results of the hadEFrac sensitivity study. Choose number
% of hadEFrac bins that optimises sensitivity considering memory.
When deciding the on number of \hadefrac{} quantiles that will be used
for the final analysis of the data both the improvement to
the sensitivity and the increase in the memory footprint due to making
the fit must be considered. With each additional \hadefrac{} bin the
memory and time needed to produce the fit almost doubles. 
The largest gain in sensitivity at all three best-fit points comes
from the initial introduction of splitting events into two \hadefrac{}
quantiles. Further improvements are seen when using three and four
quantiles. After four quantiles the improvement seen with each
increment is substantially reduced. Considering both the improvements
in sensitivity and also the increase in memory use the division of
neutrino events into 4 quantiles of \hadefrac{} is chosen. 


% Discuss hadEFrac migration due to systematics. Point to DocDB entry
% with final word on migration
The above described method involves separating events by \hadefrac{}.
It is important that events are able to migrate across \hadefrac{}
bins if an appropriate systematic shift is applied. 
Without this ability, events will be artificially constrained to
particular \hadefrac{} bins and systematic shifts will be over
constrained.
A study~\cite{hadEFracMigration} showed that events can migrate from one
\hadefrac{} bin to another when a systematic shift is applied to the hadronic
energy.


%Plot: FD hadronic energy fraction vs. neutrino energy with quantile
%boundaries 
% made with: makeHadEFracHists.C
\begin{figure}
  \centering
  \includegraphics[width=0.9\textwidth]{../../../Analysis/thesisPlots/anaImprovs/hadEFrac/hadEFracHists/cperiod1.pdf}
  \caption{Hadronic energy fraction vs. reconstructed neutrino
    energy in the far detector MC during running period 1. The
    quantile boundaries are shown for each neutrino energy bin for
    the choice of 4 hadronic energy fraction bins. } 
  \label{fig:hadEFracHistP1}
\end{figure}
\begin{figure}
  \centering
  \includegraphics[width=0.9\textwidth]{../../../Analysis/thesisPlots/anaImprovs/hadEFrac/hadEFracHists/cperiod2.pdf}
  \caption{Hadronic energy fraction vs. reconstructed neutrino
    energy in the far detector MC during running period 2. 
    The quantile boundaries are shown for each neutrino energy bin
    for the choice of 4 hadronic energy fraction bins.
  } 
  \label{fig:hadEFracHistP2}
\end{figure}
\begin{figure}
  \centering
  \includegraphics[width=0.9\textwidth]{../../../Analysis/thesisPlots/anaImprovs/hadEFrac/hadEFracHists/cepoch3b.pdf}
  \caption{Hadronic energy fraction vs. reconstructed neutrino
    energy in the far detector MC during running epoch 3b.
    The quantile boundaries are shown for each neutrino energy bin
    for the choice of 4 hadronic energy fraction bins.  } 
  \label{fig:hadEFracHistE3b}
\end{figure}
\begin{figure}
  \centering
  \includegraphics[width=0.9\textwidth]{../../../Analysis/thesisPlots/anaImprovs/hadEFrac/hadEFracHists/cepoch3c.pdf}
  \caption{Hadronic energy fraction vs. reconstructed neutrino
    energy in the far detector MC during running epoch 3c. 
    The quantile boundaries are shown for each neutrino energy bin
    for the choice of 4 hadronic energy fraction bins.  } 
  \label{fig:hadEFracHistE3c}
\end{figure}



%Plot: FD oscillated neutrino energy spectrum for each quantile

%Plot: FD hadEFrac quantiles neutrino energy with cosmics

%Plot: FD hadEFrac quantiles neutrino energy with beam bkg

%Plot: contours for 1-8 hadEFrac bins
%Plot: SA BF
\begin{figure}
  \centering
\includegraphics[width=0.9\textwidth]{../../../Analysis/thesisPlots/anaImprovs/hadEFrac/hadEFracContours/contours_stdBinning_stdSel_NOvABF.pdf}
  \caption{Sensitivity contours for the $\nu_\mu$ disappearance analysis at
    NOvA's second analysis best-fit. The sensitivity of the standard
    analysis is shown by the dotted light-blue contour. A set of analysis
    where the events are split 2, 4, 6 and 8 quantiles of \hadefrac{}
    are shown by the solid blue, solid red, dashed green and dotted
    purple contours respectively.} 
  \label{fig:hadEFracContour_NOvA}
\end{figure}

\begin{figure}
  \centering
  \includegraphics[width=0.9\textwidth]{../../../Analysis/thesisPlots/anaImprovs/hadEFrac/hadEFracContours/rej_stdBinning_stdSel_NOvABF.pdf}
  \caption{Significance of maximal mixing ($\sin^2\theta_{23} = 0.5$)
    rejection vs. the number of \hadefrac{} quantiles used in the
    analysis. The standard sensitivity, where the events are not
    divided into quantiles of \hadefrac{}, is shown by the point where quantiles = 0.} 
  \label{fig:hadEFracRej_NOvA}
\end{figure}

%Plot: MINOS BF
\begin{figure}
  \centering
\includegraphics[width=0.9\textwidth]{../../../Analysis/thesisPlots/anaImprovs/hadEFrac/hadEFracContours/contours_stdBinning_stdSel_MINOSBF.pdf}
  \caption{Sensitivity contours for the $\nu_\mu$ disappearance analysis at
    MINOS's latest best-fit. The sensitivity of the standard
    analysis is shown by the dotted light-blue contour. A set of analysis
    where the events are split 2, 4, 6 and 8 quantiles of \hadefrac{}
    are shown by the solid blue, solid red, dashed green and dotted
    purple contours respectively.} 
  \label{fig:hadEFracContour_MINOS}
\end{figure}

\begin{figure}
  \centering
  \includegraphics[width=0.9\textwidth]{../../../Analysis/thesisPlots/anaImprovs/hadEFrac/hadEFracContours/rej_stdBinning_stdSel_MINOS.pdf}
  \caption{Significance of maximal mixing ($\sin^2\theta_{23} = 0.5$)
    rejection vs. the number of \hadefrac{} quantiles used in the
    analysis. 
    The standard sensitivity, where the events are not
    divided into quantiles of \hadefrac{}, is shown by the point where quantiles = 0.
  } 
  \label{fig:hadEFracRej_MINOS}
\end{figure}

%Plot: T2K BF
\begin{figure}
  \centering
\includegraphics[width=0.9\textwidth]{../../../Analysis/thesisPlots/anaImprovs/hadEFrac/hadEFracContours/contours_stdBinning_stdSel_T2KBF.pdf}
  \caption{Sensitivity contours for the $\nu_\mu$ disappearance analysis at
    T2K's latest best-fit. The sensitivity of the standard
    analysis is shown by the dotted light-blue contour. A set of analysis
    where the events are split 2, 4, 6 and 8 quantiles of \hadefrac{}
    are shown by the solid blue, solid red, dashed green and dotted 
    purple contours respectively.} 
  \label{fig:hadEFracContour_T2K}
\end{figure}

\begin{figure}
  \centering
  \includegraphics[width=0.9\textwidth]{../../../Analysis/thesisPlots/anaImprovs/hadEFrac/hadEFracContours/rej_stdBinning_stdSel_T2K.pdf}
  \caption{Significance of rejecting ($\sin^2\theta_{23} = 0.6$)
    vs. the number of \hadefrac{} quantiles used in the
    analysis. 
    The standard sensitivity where the events are not
    divided into quantiles is shown by the point where quantiles = 0.    
  } 
  \label{fig:hadEFracRej_T2K}
\end{figure}

%Plot: rejection of max mixing vs. number of bins



\section{Optimising Neutrino Energy Binning}

The neutrino energy binning used for the muon neutrino
disappearance analysis presented in the first and second papers
consists of 0.25~GeV width bins from 0 to 5~GeV. 
Most of the neutrino oscillation information is gained between 1 and
3~GeV, in particular between 1 and 2~GeV. 
If it is wider than the neutrino energy resolution, the coarser
binning may conceal information useful in distinguishing 
between predictions made with different sets of oscillation
parameters~\cite{Marshall}. 
Therefore, using finer binning could enhance the sensitivity of the
disappearance analysis.  
The advantages of finer binning will diminish as the bin size
approaches and goes beyond the neutrino energy resolution.
Another point to consider when adjusting the neutrino energy binning
is that the number of neutrino energy bins almost
proportionally impacts the memory required for the fit. 
For this reason, the strategies presented below focus on increasing
the number of bins in the region of maximum oscillation between 1 and
3~GeV.


To investigate possible enhancements to the sensitivity, three
alternative neutrino energy binning strategies are compared with the
standard analysis neutrino energy binning. An overview of the standard
binning and the three alternative binning strategies is presented in
Table~\ref{tab:binningtable}.  
The table shows the general idea of the new binning strategies (scheme
``A'', ``B'' and ``C'') where finer bins are used in the region of
maximum oscillation (1-2~GeV) and coarser bins are used in the region
where there is less oscillation information (0-1~GeV and 3-5~GeV).
As shown in the table, the new binning strategies share a common
binning between 0 and 1~GeV of two bins, one from 0 to 0.75~GeV and
the other from 0.75 to 1~GeV. 
They also share a common binning between 3 and 5~GeV,
consisting of two 0.5~GeV bins from 3 to 4~GeV and a single 1~GeV bin
from 4 to 5~GeV.
The differences between the binning strategies occur between 1 and
3~GeV. Binning scheme ``A'' implements 80 0.025~GeV bins between 1 and
3~GeV.  
Binning scheme ``B'' implements 40 0.025~GeV bins between 1 and 2~GeV
and 4 0.25~GeV between 2 and 3~GeV. 
Finally, binning scheme ``C'' implements 10 0.1~GeV bins between 1 and
2~GeV and 4 0.25~GeV between 2 and 3~GeV. Note that binning scheme
``C'' requires less bins in total than the standard binning whilst using finer
binning in the region of interest.

%Introduce the contour plots
Comparisons of the four binning strategies in terms of sensitivity to
$\Delta m^2_{32}$ and $\sin^22\theta_{23}$ and rejection of maximal
mixing are shown for the NOvA, T2K and MINOS best fit points in
Figures~\ref{fig:binStratContour_NOvA}, \ref{fig:hadEFracRej_NOvA},
\ref{fig:binStratContour_T2K}, \ref{fig:hadEFracRej_T2K},
\ref{fig:binStratContour_MINOS} and \ref{fig:hadEFracRej_MINOS}
respectively. The figures show that the sensitivity to reject maximal
mixing increases when increasing number of neutrino energy bins (going
from std. through to ``A'' binning) for all three best fit points. The
largest increase in the sensitivity to reject maximal mixing comes
when going from the standard binning to the ``C'' binning. 
Increasing the number of bins by going from ``C'' to 
``B'' binning shows a modest improvement but requires more than double the
number of neutrino energy bins. Further, binning strategy ``A'' shows
only marginal improvement over binning strategy ``B'' whilst almost
doubling the number of neutrino energy bins. 
Binning strategy ``C'' is chosen over the other strategies as it shows
a significant improvement in sensitivity (only marginally beaten by
``A'' and ``B'') whilst reducing the number of neutrino energy bins
compared to the standard binning.


\begin{table}
  \centering
  \begin{tabular}{ | c | c | c | c | c |}
    \hline
    & \multicolumn{4}{ |c| }{Neutrino Energy Binning Scheme}\\
    \hline
    Neutrino Energy Range (GeV) & Std. & scheme A & scheme B & scheme C \\
    \hline
    0 - 0.75 & 3 & 1 & 1 & 1 \\
    0.75 - 1 & 1 & 1 & 1 & 1 \\
    1 - 2      & 4 & 40 & 40 & 10 \\
    2 - 3      & 4 & 40 & 4 & 4 \\
    3 - 4      & 4 & 2 & 2 & 2 \\
    4 - 5      & 4 & 1 & 1 & 1 \\
    \hline
    0 - 5      & 20 & 85 & 49 & 19 \\
    \hline
  \end{tabular}
  \caption{The number of neutrino energy bins per energy range for
    each neutrino energy binning scheme. The total number of neutrino
    energy bins for each scheme is shown in the last row.}
  \label{tab:binningtable}
\end{table}


% NOvA
\begin{figure}
  \centering
  \includegraphics[width=0.9\textwidth]{../../../Analysis/thesisPlots/anaImprovs/binStrat/contours/contours_binStrat_stdSel_NOvABF.pdf}
  \caption{
    Sensitivity contours for the $\nu_\mu$ disappearance analysis at
    NOvA's second analysis best-fit. The sensitivity of the standard
    analysis is shown by the dotted pink contour. A set of analysis
    where the the neutrino energy is binned according to scheme's A, B
    or C (see Table~\ref{tab:binningtable}) are shown by the sold blue,
    solid red and dotted green contours respectively.
  } 
  \label{fig:binStratContour_NOvA}
\end{figure}
\begin{figure}
  \centering
  \includegraphics[width=0.9\textwidth]{../../../Analysis/thesisPlots/anaImprovs/binStrat/contours/rej_binStrat_stdSel_NOvABF.pdf}
  \caption{
    Significance of maximal mixing ($\sin^2\theta_{23} = 0.5$)
    rejection vs. the neutrino energy binning strategy used in the
    analysis. A breakdown of the each neutrino energy binning scheme
    is shown in Table~\ref{tab:binningtable}.
  }
  \label{fig:hadEFracRej_NOvA}
\end{figure}

% T2K
\begin{figure}
  \centering
  \includegraphics[width=0.9\textwidth]{../../../Analysis/thesisPlots/anaImprovs/binStrat/contours/contours_binStrat_stdSel_T2KBF.pdf}
  \caption{
    Sensitivity contours for the $\nu_\mu$ disappearance analysis at
    T2K's latest best-fit. The sensitivity of the standard
    analysis is shown by the dotted pink contour. A set of analysis
    where the the neutrino energy is binned according to scheme's A, B
    or C (see Table~\ref{tab:binningtable}) are shown by the sold blue,
    solid red and dotted green contours respectively.
  } 
  \label{fig:binStratContour_T2K}
\end{figure}
\begin{figure}
  \centering
  \includegraphics[width=0.9\textwidth]{../../../Analysis/thesisPlots/anaImprovs/binStrat/contours/rej_binStrat_stdSel_T2KBF.pdf}
  \caption{
    Significance of rejecting ($\sin^2\theta_{23} = 0.6$) vs. the
    neutrino energy binning strategy used in the 
    analysis. A breakdown of the each neutrino energy binning scheme
    is shown in Table~\ref{tab:binningtable}.
  } 
  \label{fig:hadEFracRej_T2K}
\end{figure}

% MINOS
\begin{figure}
  \centering
  \includegraphics[width=0.9\textwidth]{../../../Analysis/thesisPlots/anaImprovs/binStrat/contours/contours_binStrat_stdSel_MINOSBF.pdf}
  \caption{
    Sensitivity contours for the $\nu_\mu$ disappearance analysis at
    MINOS's latest best-fit. The sensitivity of the standard
    analysis is shown by the dotted pink contour. A set of analysis
    where the the neutrino energy is binned according to scheme's A, B
    or C (see Table~\ref{tab:binningtable}) are shown by the sold
    blue, solid red and dotted green contours respectively.
  } 
  \label{fig:binStratContour_MINOS}
\end{figure}
\begin{figure}
  \centering
  \includegraphics[width=0.9\textwidth]{../../../Analysis/thesisPlots/anaImprovs/binStrat/contours/rej_binStrat_stdSel_MINOSBF.pdf}
  \caption{
    Significance of maximal mixing ($\sin^2\theta_{23} = 0.5$)
    rejection vs. the neutrino energy binning strategy used in the
    analysis. A breakdown of the each neutrino energy binning scheme
    is shown in Table~\ref{tab:binningtable}.
  } 
  \label{fig:hadEFracRej_MINOS}
\end{figure}



\section{Hybrid Selection Using ReMId and CVN}

% introduce possible hybrid selection using remid and cvn
The muon neutrino selection algorithms, CVN and ReMId, mentioned above
could be used in unison to improve the sensitivity to muon neutrino
disappearance. Using the selection algorithms in unison could allow
the cuts to be loosened gaining signal events without letting through
too many background events. 
When reducing the cut threshold of either remid or CVN neutral current
background events will begin to increase. A study of the
simultaneous cut levels and the resulting sensitivity will be
presented in this section. 



\section{All Analysis Improvements Combined}
\textcolor{red}{Show contours and rej. plots with individually
  optimised improvements combined }

\section{Combined Optimisation of Analysis Improvements}
\textcolor{red}{Then respin number of HadEFrac. bins again w/ energy
  bins. } 



\section{Systematic Uncertainty}

\textcolor{red}{show systematic tables after each improvement individually}



\section{Hadronic Energy Smearing Systematic}

\textcolor{red}{Systematic that smears the hadronic energy
  distributions in the near 
and far detector. Assess the effect of this systematic before and
after the improvements (just hadEFrac binning and then with all
improvements).}


