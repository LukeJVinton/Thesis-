

Three techniques with potential to improve the sensitivity of the NOvA
experiment to the muon neutrino disappearance oscillation parameters
are analysed in the following chapter. 
The first technique involves seperating neutrino events into bins of
energy resolution. 
In the second technique the number of neutrino energy bins is varied
to find the optimum number of bins.
Finally, the third technique involves a hybrid selection of the ReMId
(as used in NOvA's first and second muon neutrino disappearance
papers) and CVN selections.

The impact of each technique will be mesured using three metrics: the
sensitivity to reject maximal mixing for the NOvA second analysis
$\nu_\mu$ disappearance best fit, the sensitivity to reject maximal
mixing for the most recent MINOS $\nu_\mu$ disappearance best fit and
finally the sensitivity to reject the point ($\Delta m^2_{32} = 2.5
\times 10^{-3}~\text{eV}^2$, $sin^22\theta_{23} = 0.4$) for the most
recent T2K $\nu_\mu$ disappearance best fit.

MINOS best fit from this paper~\cite{adamson2014combined}. Measured
$|\Delta m^2_{32}| = [2.28 - 2.46] \times 10^{-3}~\text{eV}^2~(68\%
C.L.)$ and $\sin^2\theta_{23} = 0.35–0.65 \text{(90\% C.L.)}$. The
following analysis uses the best fit point for comparisons. The best
fit point is taken to be $|\Delta m^2_{32}| = 2.36
\times10^{-3}~\text{eV}^2$ and $\sin^2\theta_{23} = 0.41$. 

The T2K best fit point here refers to the results from this
paper~\cite{abe2015measurements}. The reported measurements are
$\sin^2\theta_{23}=0.528^{+0.055}_{-0.038}$ and $|\Delta
m^2_{32}|=(2.51 \pm 0.11)\times10^{-3}~\text{eV}^2)$

NOvA's second analysis best fit point is that presented in this
paper~\cite{NOvASA}. The normal ordering, lower ocant best fit point is 
$\sin^2\theta_{23} = 0.404^{+0.030}_{-0.022}$ and $|\Delta
m^2_{32}|=(2.67 \pm 0.11)\times10^{-3}~\text{eV}^2)$. 

Three (x2, for dm and sin) metrics for each improvement: SA BF
(max-mix rej.), MINOS (max-mix rej.), and T2K BF (limit and rej. of
(2.5,0.4)) 

\section{Hadronic Energy Fraction Binning}

The first sensitivity improvement considered is to seperate events by
reconstructed neutrino energy resolution. 
Events with better energy resolution are less likely to migrate
accross neutrino energy bin boundaries and into the oscillation dip.
The reconstructed neutroino energy is the sum of the reconstructed
muon energy and the reconstructed shower energy.
In the NOvA detectors, muon energy is estimated using the length of the
muon track with a resolution of 2\% while hadronic energy is estimated
from the calorimetric response of the detector to a cluster of
cell-hits to about 5\%.
Neutrino events with a larger proportion of hadronic activity will be
less well resolved by the detector.
The metric used to estimate neutrino energy resolution is the hadronic
energy fraction, $E_{had.} / E_{\nu}$. 

The $E_{had.} / E_{\nu}$ distribution is obtained for each bin of
reconstructed neutrino energy and the quantiles of $E_{had.} /
E_{\nu}$ are identified. 
These quantiles are then used as the
boundaries of the $E_{had.} / E_\nu$ bins. In this analysis the number
of \hadefrac{} bins can be varied to optimise the sensitivity. 
Figures~\ref{fig:hadEFracHistP1}, \ref{fig:hadEFracHistP2}, \ref{fig:hadEFracHistE3b}
and \ref{fig:hadEFracHistE3c} show the distribution of
hadronic energy fraction vs. reconstructed neutrino energy for each
far detector running period used in this analysis. The quantile
boundaries for a division into four hadronic energy fraction bins are
shown by the black histograms overlayed on the colour plot. 
In this case, with four \hadefrac{} bins, the lowest quantile contains
25\% of the sample with the best neutrino energy resolution. On the
other hand, the highest quantile contains the 25\% of the sample with
the worst neutrino energy resolution.


%Plot: FD hadronic energy fraction vs. neutrino energy with quantile
%boundaries 
% made with: makeHadEFracHists.C
\begin{figure}
  \centering
  \includegraphics[width=0.9\textwidth]{../../../Analysis/thesisPlots/anaImprovs/hadEFrac/hadEFracHists/cperiod1.pdf}
  \caption{Hadronic energy fraction vs. reconstructed neutrino
    energy in the far detector MC during running period 1. The
    quantile bounadaries are shown for the choice of 4 hadronic energy
    fraction bins. } 
  \label{fig:hadEFracHistP1}
\end{figure}
\begin{figure}
  \centering
  \includegraphics[width=0.9\textwidth]{../../../Analysis/thesisPlots/anaImprovs/hadEFrac/hadEFracHists/cperiod2.pdf}
  \caption{Hadronic energy fraction vs. reconstructed neutrino
    energy in the far detector MC during running period 2. The
    quantile bounadaries are shown for the choice of 4 hadronic energy
    fraction bins. } 
  \label{fig:hadEFracHistP2}
\end{figure}
\begin{figure}
  \centering
  \includegraphics[width=0.9\textwidth]{../../../Analysis/thesisPlots/anaImprovs/hadEFrac/hadEFracHists/cepoch3b.pdf}
  \caption{Hadronic energy fraction vs. reconstructed neutrino
    energy in the far detector MC during running epoch 3b. The
    quantile bounadaries are shown for the choice of 4 hadronic energy
    fraction bins. } 
  \label{fig:hadEFracHistE3b}
\end{figure}
\begin{figure}
  \centering
  \includegraphics[width=0.9\textwidth]{../../../Analysis/thesisPlots/anaImprovs/hadEFrac/hadEFracHists/cepoch3c.pdf}
  \caption{Hadronic energy fraction vs. reconstructed neutrino
    energy in the far detector MC during running epoch 3c. The
    quantile bounadaries are shown for the choice of 4 hadronic energy
    fraction bins. } 
  \label{fig:hadEFracHistE3c}
\end{figure}



%Plot: FD oscillated neutrino energy spectrum for each quantile

%Plot: FD hadEFrac quantiles neutrino energy with cosmics

%Plot: FD hadEFrac quantiles neutrino energy with beam bkg

%Plot: contours for 1-8 hadEFrac bins
%Plot: SA BF
%Plot: MINOS BF
%Plot: T2K BF


%Plot: rejection of max mixing vs. number of bins


\section{Optimising Neutrino Energy Binning}

The neutrino energy binning used for the muon neutrino
disappearance analysis presented in first and second papers consists
of 0.25~GeV width bins from 0 to 5~GeV. 
Most of the neutrino oscillation information is gained between 1 and
2~GeV.
Using finer binning in this region could enhance the sensitivity of
the disappearance analysis. 
The advantages of finer binning will diminish as the bin size
approaches the neutrino energy resolution.

To investigate possible enhancements to the sensitivity, three
alternative binning strategies are compared with the standard
analysis. 
All three binning strategies share a common binning between 0 and
1~GeV of two bins, one from 0 to 0.75~GeV and the other from 0.75 to
1~GeV. 
They also share a common binning between 3 and 5~GeV,
consisting of two 0.5~GeV bins from 3 to 4~GeV and 1 1~GeV bin from 4
to 5~GeV.
The first binning strategy uses 80 0.025~GeV bins between 1 and
3~GeV. 
The second strategy uses 40 0.025~GeV bins between 1 and 2~GeV and 4
0.25~GeV between 2 and 3~GeV. 
Finally, the third strategy uses 10 0.1~GeV bins between 1 and 2~GeV
and 4 0.25~GeV between 2 and 3~GeV.


\section{Hybrid Selection Using ReMId and CVN}


\section{All Analysis Improvements Combined}

Then respin number of HadEFrac. bins again w/ energy bins and
hybrid. Iterate until while improvements are shown.  


\section{Systematic Uncertainty}

show systematic tables after each improvement individually


\section{Hadronic Energy Smearing Systematic}

Systematic that smears the hadronic energy distributions in the near
and far detector. Assess the effect of this systematic before and
after the improvements (just hadEFrac binning and then with all
improvements).


