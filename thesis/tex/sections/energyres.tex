

\section{Energy Resolution Binning}

Energy resolution binning was implemented in MINOS to improve the
sensitivity of the experiment. Techniques similar to those found in
\cite{Marshall} will be used in the following chapter to improve the
sensitivity of the NOvA experiment. 


Table~\ref{tab:systShifts} shows the shifts in the values of
$\sin^22\theta_{23}$ and $\Delta m^2_{32}$ due to the systematic
uncertainties. 


%%%%%%%%%%%%%%%%%%%%%%%%%%%%%%%%
%%%%%%%%%% Change in width %%%%%%%%%%
\begin{table*}[t]
\caption{
\textcolor{red}{\textbf{Following copied from the NuMu SA paper: }
Sources of uncertainty and their estimated average impact on
$\sin^2\theta_{23}$ and $\Delta m^2_{32}$. Systematic uncertainties
are included in a fit to simulated data one at a time via their
associated penalty terms. The increase in the one-dimension
68\%~C.L. interval relative to when only statistical fluctuations are
included in the fit is used to estimate the average impact of
individual systematic uncertainties. The estimate is obtained by
subtracting the 68\%~C.L. intervals in quadrature, except for the
effect of $\delta_{cp}$, where the absolute difference in the size of
the intervals is used. The total impact of all sources of systematic
uncertainty is obtained by including all systematics in the fit
simultaneously, and then adding the effect of $\delta_{cp}$. Simulated
data were oscillated with $\Delta
m^2_{32}=2.66\mathord{\times}10^{-3}\text{~eV}^{2}$ and
$\sin^2\theta_{23}=0.626$.}   
}
\begin{tabular}{c c c}
\hline 
\multirow{2}{*}{Source of uncertainty} & Uncertainty in & Uncertainty in \\
& $\sin^2\!\theta_{23} (\times 10^{-3})$ & $\Delta m^2_{32}
                                           \left(10^{-6}\text{
                                           eV}^{2}\right)$ \\ 
\hline 
Normalization & +5 /  -5 & +4 / -8 \\
Absolute muon energy scale & +9 /  -8 & +3 /  -10\\
Relative muon energy scale & +9 /  -9 & +23 /  -14\\
Absolute hadronic energy scale & +5 /  -5 & +7 /  -3\\
Relative hadronic energy scale & +10 / -11 & +29 /  -19\\
Cross sections and final state interactions & +3 /  -3  & +12 /  -15 \\ 
$\delta_{cp}$ $(0 - 2\pi)$ & +0.2 / -0.3 & +10 /  -9 \\
Beam background normalization & +3 /  -6 & +10 /  -16 \\ 
Scintillation model & +4 /  -3   & +2 / -5 \\
\hline 
Total systematic uncertainty & +17 / -19 & +50 / -47 \\
\hline
Statistical uncertainty & +21 /  -23 & +93 /  -99  \\
\hline
\end{tabular}
\label{tab:systShifts}
\end{table*}


\begin{table*}[t]
  \centering
  \caption{
    Table of uncertainty in $\sin^2\!\theta_{23}$ and $\Delta
    m^2_{32}$ due to each source of systematic uncertainty when
    using the standard second analysis. This should be a break down of
    the table above.
  }
  \begin{tabular}{c c c}
    \hline 
    \multirow{2}{*}{Source of uncertainty} & 
                                             Uncertainty in & 
                                                              Uncertainty in \\
                                           & $\sin^2\!\theta_{23}
                                             (\times 10^{-3})$ & 
                                                                 $\Delta
                                                                 m^2_{32}
                                                                 \left(10^{-6}\text{
                                                                 eV}^{2}\right)$\\
    \hline 
    \textbf{Normalisation} & & \\
    norm & +0 / -0.076 & +0.2 / -0\\
    relNorm & +3.6 / -3.1 & +3.5 / -6.4\\
    % \hline
    % normalizations & +3.6 / -3.1 & +3.5 / -6.4\\
    \textbf{Muon energy scale} & & \\
    SAMuEScale & +8.5 / -7.9 & +2.8 / -10\\
    FDSAMuEScale & +8.7 / -9.2 & +23 / -14\\
    \hline
    muon E scales & +11 / -12 & +23 / -20\\
    \textbf{Hadronic energy scale} & & \\
    SACalibXY & +5 / -4.7 & +4.4 / -1.1\\
    SACalibYFunc & +0.11 / -0.076 & +2.4 / -2.9\\
    calibrations & +5 / -4.7 & +5.1 / -3.2\\
    SARelHadE & +9.9 / -11 & +29 / -19\\
    \textbf{Cross sections and final state interactions} & & \\
    TransportPlusNA49 & +0.97 / -1.9 & +3.7 / -7.1\\
    mecScale & +0.9 / -0.52 & +5.5 / -7.1\\
    RPA & +0.68 / -0.55 & +1.5 / -2.1\\
    numuSumSmallGENIE & +0.38 / -0.89 & +1 / -2.4\\
    MaNCEL & +0.048 / -0.12 & +0.28 / -0.51\\
    NormCCQE & +0.41 / -0.42 & +1.6 / -1.2\\
    MaCCQEshape & +0.64 / -0.49 & +1.7 / -2.4\\
    MaCCRES & +2.2 / -1.5 & +8.9 / -12\\
    MvCCRES & +1.4 / -1 & +5.4 / -7.1\\
    MaNCRES & +0.4 / -0.82 & +1.5 / -2.7\\
    MvNCRES & +0.095 / -0.21 & +0.4 / -0.68\\
    CCQEPauliSupViaKF & +0.91 / -0.75 & +1.4 / -2.5\\
    \hline
    GENIE+MEC+RPA & +3.1 / -2.5 & +12 / -16\\
    \textbf{Beam background normalization} & & \\
    numuNCScale & +3.3 / -6.5 & +9.3 / -17\\
    \textbf{Scintillation model} & & \\
    SABirks & +3.7 / -3.2 & +2.9 / -6.5\\
    \hline 
    osc par & +3.1 / -4 & +74 / -2.1\\
    \hline
    all systs & +15 / -18 & +87 / -39\\
    \hline
  \end{tabular}
  \label{tab:systShifts}
\end{table*}




\begin{table*}[t]
  \centering
  \caption{
    Table of uncertainty in $\sin^2\!\theta_{23}$ and $\Delta
    m^2_{32}$ due to each source of systematic uncertainty when
    splitting the sample into four hadronic energy fraction quantiles
    and a CVN -  remid hybrid selection.
  }
  \begin{tabular}{c c c}
    \hline 
    \multirow{2}{*}{Source of uncertainty} & 
                                             Uncertainty in & 
                                                              Uncertainty in \\
                                           & $\sin^2\!\theta_{23}
                                             (\times 10^{-3})$ & 
                                                                 $\Delta
                                                                 m^2_{32}
                                                                 \left(10^{-6}\text{
                                                                 eV}^{2}\right)$\\
    \hline 
    \textbf{Normalisation} & & \\
    norm & +0.0455 / -0.0503 & +0 / -0.216\\
    relNorm & +5.11 / -5.1 & +0.743 / -9.31\\
    \textbf{Muon energy scale} & & \\
    SAMuEScale & +2.46 / -1.91 & +15.3 / -22.9\\
    FDSAMuEScale & +3.1 / -3.35 & +8.01 / -3.69\\
    muon E scales & +3.82 / -3.79 & +17.9 / -23.6\\
    \textbf{Hadronic energy scale} & & \\
    SACalibXY & +1.53 / -2.09 & +14.7 / -17.5\\
    SACalibYFunc & +1.3 / -1.67 & +7.96 / -7.22\\
    calibrations & +1.92 / -2.6 & +16.2 / -18.5\\    
    SARelHadE & +3.15 / -3.39 & +8.8 / -4.76\\
    \textbf{Cross sections and final state interactions} & & \\
    numuSumSmallGENIE & +0.213 / -0.41 & +0.455 / -0.68\\
    MaNCEL & +0 / -0 & +0 / -0.216\\
    NormCCQE & +0.726 / -0.95 & +2.91 / -2.55\\
    MaCCQEshape & +0.528 / -0.426 & +1.6 / -2.92\\
    MaCCRES & +0.706 / -0.143 & +11.2 / -14.3\\
    MvCCRES & +0.566 / -0.189 & +7.22 / -9.44\\
    MaNCRES & +0.151 / -0.286 & +0.616 / -0.833\\
    MvNCRES & +0.0455 / -0.0714 & +0.186 / -0.305\\
    CCQEPauliSupViaKF & +0.573 / -0.437 & +1.82 / -3.36\\
    TransportPlusNA49 & +0.704 / -1.19 & +3.14 / -4.25\\
    mecScale & +0.582 / -1.08 & +8.97 / -9.7\\
    RPA & +0.298 / -0.16 & +2.53 / -3.2\\
    GENIE+MEC+RPA & +1.64 / -1.7 & +14.2 / -17.7\\
    \textbf{Beam background normalization} & & \\
    numuNCScale & +1.43 / -2.44 & +3.67 / -4.23\\
    \textbf{Scintillation model} & & \\
    SABirks & +0.193 / -0.739 & +12.1 / -14.2\\
    \textbf
    osc par & +2.79 / -4.05 & +69.7 / -5.23\\
    \hline
    all systs & +8.51 / -9.46 & +78.8 / -37.8\\
    \hline
  \end{tabular}
  \label{tab:systShifts}
\end{table*}

%Legend: in sin (10^-3) , dM (10^-6)




%%%%%%%%%% CPied %%%%%%%%%%          
%%%%%%%%%% CPied %%%%%%%%%%          



%%%%%%%%%%%%%%%%%%%%%%%%%%%%%%%%%%%%
%%%%%%%%%% Shift in best-fit point %%%%%%%%%%
% \begin{table*}[t]
%   \centering
%   \caption{
%     All improvs table of best fit point shifts
%   }
%   \begin{tabular}{c c c}
%     \hline 
%     \multirow{2}{*}{Source of uncertainty} & 
%                                              Shift in & 
%                                                               Shift in \\
%                                            & $\sin^2\!\theta_{23}
%                                              (\times 10^{-3})$ & 
%                                                                  $\Delta
%                                                                  m^2_{32}
%                                                                  \left(10^{-6}\text{
%                                                                  eV}^{2}\right)$\\
%     \hline 
%     \textbf{Normalisation} & & \\
%     norm& +0.03 / -0.03& +0.04 / -0.04 \\
%     relNorm& +5 / -6 & +4 / -4 \\
%     normalizations& +5 / -6& +5 / -4 \\
%     \hline
%     \textbf{Absolute muon energy scale} & & \\
%     SAMuEScale& +3 / -2& +20 / -20 \\
%     \textbf{Relative muon energy scale} & & \\
%     FDSAMuEScale& +4 / -3& +8 / -6 \\
%     \textbf{Absolute hadronic energy scale} & & \\
%     SACalibXY& +1 / -3& +21 / -15 \\
%     SACalibYFunc& +1 / -1& +8 / -8 \\
%     \textbf{Relative hadronic energy scale} & & \\
%     SARelHadE& +4 / -4& +7 / -10 \\
%     \hline
%     \textbf{Cross sections and final state interactions} & & \\
%     TransportPlusNA49& +1 / -1& +5 / -2 \\
%     mecScale& +1 / -1& +10 / -11 \\
%     RPA& +0.6 / -0.0 & +6 / -0 \\
%     numuSumSmallGENIE& +0.3 / -0.3& +1 / -1 \\
%     MaNCEL& +0.01 / -0.01& +0.07 / -0.03 \\
%     NormCCQE& +0.96 / -0.80& +3.3 / -2.8 \\
%     MaCCQEshape& +0.51 / -0.54& +2.2 / -2.0 \\
%     MaCCRES& +0.67 / -0.74& +15 / -13 \\
%     MvCCRES& +0.55 / -0.29& +9.3 / -8.3 \\
%     MaNCRES& +0.24 / -0.17& +0.82 / -0.65 \\
%     MvNCRES& +0.057 / -0.047& +0.21 / -0.18 \\
%     CCQEPauliSupViaKF& +0.50 / -0.65& +2.6 / -2.3 \\
%     \hline
%     \textbf{Beam background normalization} & & \\
%     numuNCScale& +2 / -2& +4 / -4 \\
%     \textbf{Scintillation model} & & \\
%     SABirks& +0.3 / -0.3& +13 / -14 \\
%     \hline 
%   \end{tabular}
%   \label{tab:systShifts}
% \end{table*}



% \begin{table*}[t]
%   \centering
%   \caption{
%     std syst table of best fit point shifts.
%   }
%   \begin{tabular}{c c c}
%     \hline 
%     \multirow{2}{*}{Source of uncertainty} & Shift in & Shift in \\
%                                            & $\sin^2\!\theta_{23}
%                                              (\times 10^{-3})$ &
%                                                                  $\Delta
%                                                                  m^2_{32}
%                                                                  \left(10^{-6}\text{
%                                                                  eV}^{2}\right)$
%     \\
%     \hline 
%     \textbf{Normalisation} & & \\
%     norm& +0.049 / -0.049& +0.13 / -0.13 \\
%     relNorm& +3.5 / -3.6& +4.3 / -4.4 \\
%     normalizations& +3.4 / -3.6& +4.4 / -4.5 \\
%     \hline
%     \textbf{Absolute muon energy scale} & & \\
%     SAMuEScale& +9.7 / -8.8& +2.9 / -8.5 \\
%     \textbf{Relative muon energy scale} & & \\
%     FDSAMuEScale& +10 / -9.6& +26 / -16 \\
%     muon E scales& +20 / -18& +29 / -4.6 \\
%     \textbf{Absolute hadronic energy scale} & & \\
%     SACalibXY& +5.2 / -5.7& +3.4 / -13 \\
%     SACalibYFunc& +0.099 / -0.033& +2.3 / -2.8 \\
%     calibrations& +5.3 / -5.5& +0.55 / -15 \\
%     \textbf{Relative hadronic energy scale} & & \\
%     SARelHadE& +11 / -11& +21 / -32 \\
%     \hline
%     \textbf{Cross sections and final state interactions} & & \\
%     TransportPlusNA49& +1.6 / -1.1& +6.8 / -3.6 \\
%     mecScale& +0.79 / -0.84& +5.6 / -6.6 \\
%     RPA& +1.4 / -0& +3.1 / -0 \\
%     numuSumSmallGENIE& +0.58 / -0.56& +1.7 / -1.6 \\
%     MaNCEL& +0.1 / -0.046& +0.49 / -0.22 \\
%     NormCCQE& +0.46 / -0.38& +1.7 / -1.4 \\
%     MaCCQEshape& +0.6 / -0.61& +1.9 / -1.8 \\
%     MaCCRES& +2 / -2.1& +11 / -8.8 \\
%     MvCCRES& +1.3 / -1.3& +6.5 / -5.4 \\
%     MaNCRES& +0.64 / -0.46& +2.2 / -1.8 \\
%     MvNCRES& +0.15 / -0.12& +0.56 / -0.48 \\
%     CCQEPauliSupViaKF& +0.86 / -0.9& +1.6 / -1.8 \\
%     \hline
%     GENIE+MEC+RPA& +6.6 / -4.7& +15 / -8.7 \\
%     \hline
%     \textbf{Beam background normalization} & & \\
%     numuNCScale& +4.5 / -4.8& +14 / -14 \\
%     \textbf{Scintillation model} & & \\
%     SABirks& +4 / -3.7& +3.1 / -5 \\
%     \hline 
%     all systematics& +20 / -10& +12 / -31 \\
%     \hline
%   \end{tabular}
%   \label{tab:systShifts}
% \end{table*}
















