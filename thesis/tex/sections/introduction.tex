Use notes from the review paper \cite{LBNReview_Rubbia}.

Let us start in 1914, when Chadwick presented experimental
evidence~\cite{chadwick1914distribution} that the energy spectrum of
electrons emitted in $\beta$~decay was continuous instead of being
discrete as expected. This meant either $\beta$~decay was not a
two-body process or conservation of energy was violated.   
The solution arrived in 1930, when neutrinos were postulated by
Wolfgang Pauli~\cite{pauli1930letter} as a ``desperate remedy'' to the
apparent non-conservation of energy in nuclear $\beta$~decay. He
suggested that an additional neutral and extremely light
particle was produced in $\beta$~decay which carried away
the undetected energy.  
Pauli referred to this aditional particle as a ``neutron''. In 1934
Fermi~\cite{fermi1934tentativo} formulated a theory of $\beta$~decay
and re-named the additional particle as the ``neutrino''.

In 1956, Cowan and Reines~\cite{Reines:1956rs} incredibly overcame the
hurdle of detecting neutrinos~\footnote{They expressed the
  double edged sword of the validity of Pauli's neutrino proposal and
  the difficulty of neutrino detection as “the very characteristic of
  the particle which makes the proposal plausible - it’s ability to
  carry off energy and momemtum without detection”.} 
and provided the first
direct evidence for their existence. The pair setup an experiment to
measure the flux of neutrinos emitted from a nuclear reactor. The
neutrinos from the reactor were produced via $\beta$~decay and were
detected via inverse $\beta$~decay ($p + \bar{\nu} \rightarrow n +
e^+$). If a neutrino interacted within the detector it would produce a
characteristic signal of a pair of photons from electron-positron
annihilation and a delayed photon from neutron capture.

Muon and tau neutrinos were discovered several decades apart.
In 1962 neutrinos which produce muons but not electrons when
interacting with matter were
observed~\cite{MuonNeutrino:PhysRevLett.9.36}. In the paper it was
suggested that neutrinos produced in association
with a muon are muon neutrinos which are distinct
from the electron neutrinos previously observed in $\beta$~decay. 
In 1989, The ALEPH collaboration published measurements of the mass
and width of the Z~boson~\cite{ZWidth:Decamp:1989tu}. Consequently
they were able to constrain the 
number of active light neutrinos to be three at 98\% confidence level.
Final results from the ALEPH experiment provided conclusive proof that
the number of light active neutrinos is indeed three~\cite{ALEPH:2005ab}.
In 2000, the DONUT experiment reported an observation of four tau
neutrino neutrinos with a background estimation of 0.34
events~\cite{nutau:Kodama:2000mp}. This third neutrino completed the
set of the three standard model neutrinos associated with the three
charged leptons. 


The first hints of neutrino oscillations came from experiments
measuring the flux of solar neutrinos. The Fusion reactions by which energy
is produced in the sun are modelled by the Standard Solar Model. The model
predicts three major neutrino emissions: pp neutrinos (
$p + p \rightarrow d + e^+ + \nu_e$, $E_{max.}=420$~keV), $^7\text{Be}$
neutrinos ($^7\text{Be} + e^- \rightarrow ^7\text{Li} + \nu_e$, $E_{max.} =
860$~keV) and $^8\text{B}$ neutrinos ($^8\text{B} + e^- \rightarrow
^8\text{Be} + e^+ + \nu_e$, $E_{max.}
=14$~MeV).~\cite{GALLEX:Anselmann:1992um}  
In 1968 Ray Davis et. al. published the ``Search For Neutrinos From the
Sun'' paper~\cite{Davis:PhysRevLett.20.1205}. In the
same year Bahcall et. al. published a prediction for the solar neutrino
flux experienced by Ray Davis's experiment based on the standard solar
model~\cite{BAHCALL1968359}. 
Davis used the 
interaction $Cl^{37}+\nu \rightarrow e^- + Ar^{37}$ to measure the
flux of $^8\text{B}$ and $^7\text{Be}$ solar electron neutrinos.
The measured rate of 
neutrinos was found to be only one third of the rate predicted by
Bahcall. At the time of publication this discrepency was generally 
attributed to errors in either the measurement technique or the
standard solar model at high neutrino energies and became known as the
``solar neutrino puzzle''.  In 1989, the Kamiokande-II experiment
confirmed the deficit of  $^8\text{B}$ solar electron neutrinos
relative to the standard solar
model~\cite{Kamiokande:PhysRevLett.63.16}.   
In 1991 and 1992, the deficit was further
confirmed by two experiments, SAGE~\cite{SAGE:PhysRevLett.67.3332} and  
GALLEX~\cite{GALLEX:Anselmann:1992um}, which both measured the rate of  
the less energetic pp solar neutrinos. These experiments showed that the
flux deficit occured for low energy pp neutrinos as well as the higher
energy $^8\text{B}$ and $^7\text{Be}$ neutrinos measured previously.

Either the standard solar model
was incorrect or electron neutrinos produced in the Sun do not
all survive the journey to the Earth. Neutrino oscillation due to
massive neutrinos became accepted as a realistic possibility.
Previously, in the years 1962-1968, Pontecorvo, Maki, Nakagawa, and
Sakata had
formulated a theory including electron and muon neutrinos which were
able to change from one to the other~\cite{Pontecorvo:1967fh, Gribov:1968kq, Maki:1962mu}. 

In 1998, Super Kamiokande reported evidence for the oscillation of
atmospheric neutrinos~\cite{SK:PhysRevLett.81.1562}. They measured the
rate of muon and electron neutrinos originating from collisions of
cosmic rays with nuclei in the upper atmosphere. The detector allowed
the reconstruction of the direction of the incoming neutrino, 
this revealed the baseline from production to detection and whether
the neutrino travelled through the Earth.
Electron and muon
neutrinos are produced in the upper atmosphere in the ratio 1:2
(electron:muon). 
The assymetry of upward and downward going events was measured for
both electron and muon neutrinos. The electron neutrino assymetry was
consistent with zero but a momentum dependant assymetry was observed
for muon neutrinos at more than 6 standard deviations. The data was
concluded to show evidence for neutrino oscillations.

In 2005, SNO published measurements of the flux of solar $^8\text{B}$
neutrinos~\cite{SNO:PhysRevC.72.055502}. They measured
the rate of charged current electron neutrino events and were also
able to measure the rate of neutral current events due to all three
active neutrinos. The measured neutral current flux was consistent
with the flux expected from the standard solar model but the measured
flux of electron neutrinos was significantly lower than the measured
total flux of active neutrinos. Neutrino oscillations were reconfirmed
solving the solar neutrino puzzle.

Since the confirmation of neutrino oscillations 
the spotlight has moved on to
the unknown neutrino properties, such as:
the parameters which govern oscillations, the nature of
neutrinos (whether they are Dirac or Majorana particles), strength of
interaction with matter, the mass, origin of mass and
the number of neutrinos. 
Experimental and theoretical advances have made progress on the above
questions during the last $\approx$ 60 years~\cite{pdg} and the
observation of neutrino oscillations has opened new fundamental
questions regarding the origin of fermion masses and the relationship
between quarks and leptons~\cite{Mohapatra:2005wg}. 


% The importance of the properties of neutrinos and their role in the
% universe began to reveal itself~\cite{Winter:2000xh}~\cite{Kolb:1990vq}.






