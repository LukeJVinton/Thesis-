Use notes from the review paper \cite{LBNReview_Rubbia}.

Let us start in 1914, when Chadwick presented experimental
evidence~\cite{chadwick1914distribution} that the energy spectrum of
electrons emitted in $\beta$~decay was continuous instead of being
discrete as expected. This meant either $\beta$~decay was not a
two-body process or conservation of energy was violated.   
The solution arrived in 1930, when neutrinos were postulated by
Wolfgang Pauli~\cite{pauli1930letter} as a ``desperate remedy'' to the
apparent non-conservation of energy in nuclear $\beta$~decay. He
suggested that an additional neutral and extremely light
particle was produced in $\beta$~decay which carried away
the undetected energy.  
Pauli referred to this aditional particle as a ``neutron''. In 1934
Fermi~\cite{fermi1934tentativo} formulated a theory of $\beta$~decay
and re-named the additional particle as the ``neutrino''.

In 1956, Cowan and Reines~\cite{Reines:1956rs} incredibly overcame the
hurdle of detecting neutrinos~\footnote{They expressed the
  double edged sword of the validity of Pauli's neutrino proposal and
  the difficulty of neutrino detection as “the very characteristic of
  the particle which makes the proposal plausible - it’s ability to
  carry off energy and momemtum without detection”.} 
and provided the first
direct evidence for their existence. The pair setup an experiment to
measure the flux of neutrinos emitted from a nuclear reactor. The
neutrinos from the reactor were produced via $\beta$~decay and were
detected via inverse $\beta$~decay ($p + \bar{\nu} \rightarrow n +
e^+$). If a neutrino interacted within the detector it would produce a
characteristic signal of a pair of photons from electron-positron
annihilation and a delayed photon from neutron capture.

In 1962 neutrinos which produce muons but not electrons when
interacting with matter were
observed~\cite{MuonNeutrino:PhysRevLett.9.36}. In the paper it was
suggested that neutrinos produced in association
with a muon are muon neutrinos which are distinct
from the electron neutrinos previously observed in $\beta$~decay. 

%Standard Solar Model....

In 1968 Ray Davis et. al. published the ``Search For Neutrinos From the
Sun'' paper~\cite{Davis:PhysRevLett.20.1205}. They used the
interaction $Cl^{37}+\nu \rightarrow e^- + Ar^{37}$ to measure the
flux of $^8\text{B}$ and $^7\text{B}$ solar
electron neutrinos. The measured rate of 
neutrinos was found to be only one third of the rate predicted by the
standard solar model. At the time of publication this discrepency was
generally 
attributed to errors in either the measurement technique or the solar
model and became known as the solar neutrino puzzle. In 1989 the
Kamiokande-II experiment confirmed a deficit of 
$^8\text{B}$ solar electron neutrinos relative to the standard solar
model~\cite{Kamiokande:PhysRevLett.63.16}. With this confirmation, the
possibility of the solar neutrino problem being due to errors in
experimental technique was reduced. Either the standard solar model
was incorrect or electron neutrinos produced in the Sun do not
all survive the journey to the Earth. The deficit was further
confirmed by the experiments SAGE~\cite{SAGE:PhysRevLett.67.3332} and
GALLEX~\cite{GALLEX:Anselmann:1992um} which measured the rate of pp
solar neutrinos. 



In 1989, The ALEPH collaboration published measurements of the mass
and width of the Z~boson~\cite{ZWidth:Decamp:1989tu}. Consequently
they were able to constrain the 
number of active light neutrinos to be three at 98\% confidence level.
Final results from the ALEPH experiment provided conclusive proof that
the number of light active neutrinos is indeed three~\cite{ALEPH:2005ab}.

In 2000, the DONUT experiment reported an observation of four tau
neutrino neutrinos with a background estimation of 0.34
events~\cite{nutau:Kodama:2000mp}. This third neutrino completed the
set of the three standard model neutrinos associated with the three
charged leptons. 




% The importance of the properties of neutrinos and their role in the
% universe began to reveal itself~\cite{Winter:2000xh}~\cite{Kolb:1990vq}.

% What are the important question to answer regarding neutrinos? number,
% types, nature of interaction with matter, strength of interaction with
% matter, mass, and origin of mass.

Experimental and theoretical advances have made progress on the above
questions during the last $\approx$ 60 years~\cite{pdg}. Observation of
neutrino oscillations has opened new fundamental questions regarding
the origin of fermion masses and the relationship between quarks and
leptons~\cite{Mohapatra:2005wg}. 


Observation of neutrino oscillation: deficit in solar
neutrinos~\cite{Davis:PhysRevLett.20.1205}, Kamiokande, SNO etc. 


Formulation of PMNS
matrix~\cite{Pontecorvo:1967fh, Gribov:1968kq, Maki:1962mu}. 


First measurements of neutrino mixing: MINOS, SNO, Super Kamiokande,
Kamland, double chooz, RENO.



