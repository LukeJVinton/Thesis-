%Introduce the following chapter
This chapter presents an overview of the steps taken for the second
muon neutrino disappearance analysis. 
The second analysis summary~\cite{SASummary} and the second analysis
disappearance paper~\cite{NOvASA} are both refered to.



\section{Event Reconstruction}\label{sec:reco}

The reconstruction begins with a collection of above threshold APD
signals clustered in space and time~\cite{baird2015analysis,
  ester1996density}. 
The clusters of APD signals are used to reconstruct event
candidates~\cite{baird2015analysis}.

The trajectories of charged particles within the detector are
reconstructed using a technique based on the Kalman filter
algorythm~\cite{kalman1960new}~\cite{KalmanTrackNote}. An example
event with tracks found using the NOvA Kalman filter is shown in
Figure~\ref{fig:recoTrackEvd}. 
The NOvA Kalman tracker is used to estimate the true trajectory of a
particle within the detector given positions of cell
hits. Trackable particles are charecterised by trajectories with long
straight sections, dominated by small angle multiple scatttering, and
intermittent large scattering angles caused electromagnetic or
strong interactions. 
The track finding process initially starts from the downstream end of
the detector, where particles emerging from a NuMI interaction will
(on average) be the most seperated, and proceeds upstream. 
Track finding and fitting is performed seperately in each detector
view since the trajectory in each view is independent. Later the views
are matched to reconstruct the fill three dimensional tracks. This
matching of views also presents an opportunity to recover tracks
reconstructed as multiple seperate tracks due to hard
scattering.
~\cite{KalmanTrackNote}  

\begin{figure}
  \centering
  \includegraphics[width=0.7\textwidth,angle=-90]{../../img/Methodology/kalmanTrackEvd.pdf} 
  \caption{Example of reconstructed tracks found using the NOvA kalman
    tracker in the far detector simulation. Individual reconstructed
    tracks are shown by the red, blue and green lines. The x-position
    and y-position views are shown in the top and bottom half of the
    figure respectively. This figure taken from the NOvA Kalman track
    technote~\cite{KalmanTrackNote}.   
  } 
  \label{fig:recoTrackEvd}
\end{figure}


\section{Selection and Background}\label{sec:sel}

A series of selection packages are used to identify muon neutrino
events within the detector and also to reject background events.
There two main sources of background to the muon neutrino
disappearance analysis are from cosmic ray muons and beam induced
backgrounds including neutral current events, $\nu_e$ charged current
and $\nu_\tau$ charged current events. 
The good spill, data quality, cosmic, containment and reconstructed
particle identification selections are described in the following
paragraphs. 
  

% beam bkg and cosmics
The beam background events are estimated using the simulation for each
detector.
Events passing the selection but failing a truth cut
requiring a muon or anti-muon neutrino are deemed to be
background. 
The cosmic background (mostly secondary cosmic ray muons) is estimated
using two samples of far detector data which occur outside of the beam
spill window.  
The first sample is taken using the timing sidebands of the data
collected with the NuMI trigger. This sample matches the exposure of the
detector to the NuMI beam but the shape of the neutrino energy
distribution is statistically limited.
The second sample is taken using a cosmic trigger where data is
collected for activity within the detector outside of the beamspill window.

% intro. cut flow
Figure~\ref{fig:cosCutFlow} shows the estimated number of cosmic
background and signal events after each successive selection is
applied to the sample.
As shown in Figure~\ref{fig:cosCutFlow}, selections are made for good
spills, data qualtiy, cosmic rejection, containment and neutral
current rejection. 
%good spills
The first selection, at the top of Figure~\ref{fig:cosCutFlow}, is the
good spill selection which requires that the
NuMI beam is produced within accepatble bounds of spill time ($< 0.5
\times 10^9~\text{seconds}$), 
spill POT ($> 2 \times 10^{12}$), 
horn current ($-202~\text{kA} < I < -198~\text{kA}$), proton beam
position on NuMI target ($-2~\text{mm} < pos(x,y) < 2~\text{mm}$) and
beam width ($0.57~\text{mm} < width(x,y) <1.58~\text{mm}$)
\cite{NOvASADAQ}~\cite{NOvAFABeamMon}. 

% selection cutflow plot
\begin{figure}
  \centering
  \includegraphics[width=0.9\textwidth]{../../img/selection/cutflow.pdf}
  \caption{The number of signal events (red) and cosmic background
    events (yellow) surviving each successive analysis selection. The
    signal is estimated from the simulation and the cosmic background
    is estimated from the timing sidebands of the NuMI trigger. Figure
    taken from~\cite{NOvANumuSABless}.
  } 
  \label{fig:cosCutFlow}
p\end{figure}


%explain data quality
Next, the data quality selection removes events with problems in one
or more data concentrator modules. 
The selection requires that: no data concentrator modules completely
drop out during the spill, tracks do not all stop at DCM edge
boundaries (this signals that the detector is out of sync and uses a
DCM edge metric~\cite{DCMEdgeMetric}), and
\textcolor{red}{fracdcm3hits $< 0.45$. Where is this defined? Is there
  a technote?}.

% explain the cosmic BDT
The cosmic rejection selection utilises a boosted decision tree (BDT)
to create a cosmic rejection particle identification
variable~\cite{CosRejTechnote}. 
The BDT is passed 11 variables including the angle of the track
relative the NuMI beam direction, maximum height of activity within
the detector, length of the track and the number of hits.

%explain containment
A selection is made to ensure the events used for the analysis are
fully contained within the detetor. This selection serves two
purposes, it ensures all the energy from a muon neutrino charged
current interaction is deposited within the detector. The selection
also assists the rejection of cosmic rays in the far detector and rock
muons (from interactions of neutrionos in the rock upstream of the
detector) in the near detector.
The containment selection uses both the kalman tracks reconstructed as
described in Section~\ref{sec:reco} and the hits to select fully
contained events.
The hit component of the selection requires that there are no hits in
the outer two cells in either view and also no hits in the first or last
two planes of the detector.
The track component of the selection
requires that the forward or backward projection from the end or start
of the kalman track passes through at least 10 cells before exiting
the detector. 

% near detector containment
The near detector has a slightly different selection to account for
the muon catcher and the relative absence of the cosmic ray muon
background. As described in Section~\ref{sec:neardet}, the
muon catcher at the downstream end of the near detector is 2/3 the
height of the fully active detector. The differences with respect to
the far detector containment selection are outlined below.
The projection cut is loosened to at least
4 planes projected forward and at least 8 planes projected
backwards. 
The start position of the
kalman track must occur in the fully active detector upstream of the
muon catcher (z$=1150$~cm, active detector ends at z??). 
The kalman track must either have either end within the the
fully active detector or the position of the track within the detector
at transition from fully active to muon catcher must be below the
height of the muon catcher.
~\cite{SASummary}


% explain the ReMId selection
A k-nearest neighbours (kNN) classifier~\cite{altman1992introduction}
known as Reconstructed Muon Identification (ReMId)
is used to identify muon candidates among the particle
trajectories within an event~\cite{raddatzThesis}. 
The Reconstructed Muon Identification algorythm uses the following
four variables to distinguish muons from pions using the kNN
classifier:  
dE/dx, scattering, track length and fraction of plains along the track
consistent with additional hadronic energy
depositions. The kNN output is a score for each event, the
distribution of ReMId scores is shown in Figure~\ref{fig:remidDist},
the muon neutrino charged current signal events are shown by the black
histogram which peaks close to 1 and the neutral current background
events are shown by the red histogram. 
For each event passing the preceeding cuts, the particle with the
highest ReMId score is designated as the primary muon track.
Events with $\text{ReMId} > 0.75$ are taken as muon neutrino
charged current interactions, in the far detector this selection
results in a signal efficiency and purity for contained events of 81\%
and 95\% respectively. 
The impact of neutrino oscillations on the efficiency
and purity is approximately accounted for by setting $\Delta m^2_{32}
= 2.5 \times 10^{-3} \text{eV}^2$ and $\sin^2\theta_{23} =
0.5$.~\cite{raddatzThesis}~\cite{remidNote}.   



\begin{figure}
  \centering
  \includegraphics[width=0.6\textwidth,angle=-90]{../../img/Methodology/RemidDistribution.pdf}
  \caption{
    Reconstructed Muon Identification distribution for muons neutrino
    charged current events (black histogram) and neutral current
    events (red histogram). For the standard analysis events are
    required to have a ReMId score of 0.75 or greater. Figure taken
    from~\cite{remidNote}.  
  } 
  \label{fig:remidDist}
\end{figure}


% cvn selection
An alternative method called Convolutional Visual Network (CVN) for
selecting muon neutrino charged current events has been developed.
The CVN algorithm identifies muon neutrino charged current
events based on the event topology and does not require detailed event
reconstruction. 
The output from the muon neutrino charged current event classifier is
shown in Figure~\ref{fig:cvnDist}. The signal events are shown by the
blue histogram. The background neutral current, appearance $\nu_e$ and
inherent beam $\nu_e$ events are shown by the blue, purple and pink
histograms respectively.
~\cite{aurisano2016convolutional}

\begin{figure}
  \centering
  \includegraphics[width=0.6\textwidth,angle=-90]{../../img/Methodology/cvnDistribution.pdf}
  \caption{
    Output of the Convolutional Visual Network  for muon neutrino
    charged current event identification. Muon neutrino
    charged current events are shown by the green histogram. The
    backgrounds of neutral current
    events are shown by the blue histogram, appearance $\nu_e$ events
    are shown by the purple histogram and the inherent beam $\nu_e$
    events are shown by the pink histogram. Figure taken
    from~\cite{aurisano2016convolutional}.   
  } 
  \label{fig:cvnDist}
\end{figure}



\section{Calibration}\label{sec:calibration}

NOvA's energy calibration is performed in three stages.
The attenuation calibration corrects the detector response along the
NOvA cells using through going muons. The attenuation corrected
response should be uniform across the detector. 
The absolute energy calibration uses muons that stop within the
detector to find a constant to convert the attenuation correct
response into physically meaningful units of GeVs.
The drift calibration accounts and adjusts for the variatition of the
detector response with time caused by seasonal effects and instrument
degradation. 


\subsection{Attenuation and Threshold Calibration}

The attenuation calibration uses the energy depositions of
through-going cosmic ray muons to produce constants and formulae such
that such that the corrected detector response to energy depositions
is uniform across the detector. 
Attenuation fits are performed on a channel by channel basis, in other
words the detector response along the length of each NOvA cell is
fitted to produce a calibration for that cell. 

Before calibrating, the detector response is divided by the the
path-length (through a cell) to reduce the impact of variations in
path-length arising from reconstruction effieciencies of different
track angles. 
To provide an accurate path length estimation only cells with hits
from the same track in adjacent cells within the same plane are used
for the calibration. Using this selection allows the path-length
through a cell to be calculated from the cell width and the angle of
the track. 

Figure~\label{fig:attenCurvND} shows the uncalibrated
detector response along the length of a NOvA near detector cell, data
and a fit to the data are shown by the black points and the blue curve
respectively. The fit to the data is used to provide constants that
correct the detector response such that it should be uniform across
the detector.~\cite{alexAtten}~\cite{prabAtten}

\begin{figure}
  \centering
\includegraphics[width=0.6\textwidth, angle=-90]{../../img/Methodology/attenCurveNDCell.pdf}
  \caption{
    An example of a PE/cm vs. distance from cell center curve in the
    NOvA near detector. The data is shown by the black points with
    statistical erros. A fit to the data is shown by the blue
    curve. Figure taken from~\cite{alexAtten}.
  }
  \label{fig:attenCurvND}
\end{figure}


\subsection{Absolute Calibration}
% absolute calibration
The absolute calibration method is described in the first analysis
absolute calibration tech note~\cite{lukeAbsCal}. The results of the
second analysis calibration and the differences compared to the first
analysis are described in the second analysis absolute calibration
technical note~\cite{dianaAbsCal}.
The NOvA absolute calibration uses the energy deposited by stopping
muons as a standard candle. 
To reduce systematic uncertainties, only those
energy deposits in a 1-2 m window away from the muon track end point
are used. 
The mean of the detector response distribution is found for
data and MC in both near and far detectors. 
The mean of the distribution of true energy deposits in the track
window is used to provide a conversion factor between the detector
response and the true energy deposited in the scintillator for minimum
ionising muons.
Figure~\ref{fig:calibrateddEdx} shows the resulting calibrated dE/dx
distribution of stopping muons in NOvA's far
detector.~\cite{lukeAbsCal} 

\begin{figure}
  \centering
  \includegraphics[width=0.6\textwidth, angle=-90]{../../img/Methodology/absCaldEdx.pdf}
  \caption{
    Calibrated dE/dx (MeV/cm) for hits within the 100~cm track window
    in the NOvA far detector. 
    Data and simulation are shown by the black points and red
    histogram respectively.
    Figure taken from~\cite{lukeAbsCal}.
  } 
  \label{fig:calibrateddEdx}
\end{figure}


\subsection{Drift Calibration}

A drift calibration is being developed to account for variations in
the detector response with time. At the point of writing this stage of
the calibration is not yet implemented.


\subsection{Timing Calibration}

The aim of the timing calibration is to prescisely synchronise each
detector externally with the neutrino beam and internally among
the electronic detector components. 
The internal timing calibration measures and accounts for timing
offsets between data concentrator modules (see~\ref{sec:DAQ}) using
hit times from cosmic ray tracks crossing multiple data concentrator
modules.~\cite{evanTiming}



\section{Energy Reconstruction}

The neutrino energy is estimated with a resolution of aproximately 7\%
from the muon and hadronic energy. 
The muon energy is estimated with a resolution of 3.5\% using the
length of the track within the detector.  
As a consequence, the energy of muon's that do not stop within the
detectors cannot be estimated with the same accuracy. 
The visible hadronic energy is estimated from the calibrated
detector response (see Section~\ref{sec:calibration}) of hits not
associated with the muon track. 
The simulation is used to create a conversion function from visible
hadronic energy to total hadronic energy using a linear fit. 
The total hadronic energy is estimated, with a resolution of 25\%,
from the visible energy using the conversion function.~\cite{NOvASA} 


\section{Extrapolation}\label{sec:extrap}
The NOvA near detector is used to compare distributions resulting from
neutrino interactions in data and simulation. Any significant
difference between the two means that a process exists that is not
correctly modelled in the simulation. 
The differences in the near detector neutrino energy spectrum between
data and simulation are extrapolated to predict the far detector
neutrino energy spectrum. The extrapolation accounts for neutrino
oscillations, acceptance differences and flux differences between the
near and far detector. 

The extrapolation proceeds in stages. 
First, simulation estimated background is subtracted from the the near
detector data spectrum.  
A reconstructed to true neutrino energy matrix obtained from the near
detector simulation is used to convert the background subtracted
reconstructed neutrino energy into a true energy spectrum. 
A far-to-near detector event ratio is used to account for the effect
of neutrino oscillations and the different acceptances of the two
detectors.
The true neutrino energy spectrum is multiplied by the far to near
event ratio as a function of true neutrino energy to produce the far
detector true neutrino energy spectrum.  The far detector
true neutrino energy is converted back to reconstructed
neutrino energy using the far detector reconstructed to true neutrino
energy matrix obtained from the far detector simulation. 
Finally, background events due to cosmic rays (from data) and beam
backgrounds (from MC) are added to the extrapolated far detector
reconstructed neutrino energy distribution to form a prediction that
will later be compared to the far detector data.
\cite{NOvASA}



\section{GENIE Tune}\label{sec:genietune}

% introduce 2p2h MEC
Recently other experiments have presented evidence suggesting an
additional process contributing to the neutrino interaction event
rate~\cite{rodrigues2016identification}. 
The distribution of hadronic
energy in the NOvA near detector data provides supporting evidence for
the additional process~\cite{tuftsWeightNote}. 
For the second analysis the simulation was adjusted to include a
semi-empirical model of the so called two-particle two-hole (2p2h)
process where 
neutrinos scatter from nucleon pairs within the nucleus via a meson
exchange current (MEC) between the nucleons.
The model is motivated by observations of a rate enhancement in
electron on nucleus scattering where the discrepency is modelled via
the additional 2p2h process including meson exchange
currents~\cite{benhar2008inclusive}. 

% explain role of axial masses and how they are used as an approximate
% way of including unmodelled processes.
Axial mass ($\text{M}_{\text{A}}$) gets included as a bodge to the
cross section. $\text{M}_{\text{A}}$ should be 1~GeV. 
If it is greater it indicates the existence of an extra process at
that energy scale. 

% explain the CC QE cross section plot. Explain how the original MEC
% weight in genie is found from the difference between data and the
% free nucleon (M_A = 1~GeV) cross section.
Figure~\ref{fig:QEXSection_katori} shows the quasielastic cross
section vs. reconstructed neutrino energy for data taken with the
MiniBooNE (red circles) and the NOMAD (blue stars) experiments. The
MiniBooNE and NOMAD experiments measured the cross section for
neutrino interactions with reconstructed neutrino energy in the ranges
$0.4 - 2~\text{GeV}$ and $4 - 80~\text{GeV}$ respectively.
At low energy MiniBooNE data shows good agreement with the Fermi Gas
model with an axial mass of $1.35~\text{GeV}$. At higher energies
NOMAD data~\cite{lyubushkin2009study} shows good agreement with the
Fermi Gas model with an axial 
mass of $1.03~\text{GeV}$ (the free nucleon cross section). 
The NOMAD data indicates agreement with the free nucleon cross
section, implying the multinucleon process is not present at higher
energies. 
Therefore, the GENIE MEC cross section is set to vanish
above 5~GeV where data indicates agreement with the free nucleon
model. 
A cross section simply switching off is not expected for a physical
process.

\begin{figure}
  \centering
\includegraphics[width=0.95\textwidth]{../../img/Methodology/QEXSection_MiniBooNE_NOMAD.png}
  \caption{
    Charged current quasielastic cross section vs\. reconstructed
    neutrino energy for MiniBooNE (red circles) and NOMAD (blue
    stars). The solid red and dotted blue curves show the theory
    predictions based on modelling the nucleons as a fermi gas with an
    axial mass of $1.35$ and $1.03$~GeV respectively. Figure taken
    from~\cite{katori2008measurement}.
  } 
  \label{fig:QEXSection_katori}
\end{figure}


% explain the different tufts weight components: Tufts2p2hWgt,
% TuftsWeight, fixInitState&XSec

Several changes were made to fix the limitations of the semi-empirical
model described above.
Figures~\ref{fig:tuftsWeight_hadE}, \ref{fig:tuftsWeight_muE} and
\ref{fig:tuftsWeight_nuE} show the effect of the adjustments made to
the simulation on the agreement between data and MC in the near
detector. Corresponding ratios of data over MC are shown in
Figures~\ref{fig:tuftsWeight_hadERatio}, \ref{fig:tuftsWeight_muERatio} and
\ref{fig:tuftsWeight_nuERatio}.  
First of all, weighting functions were used to correct the basic
2p2h-MEC cross section behavior.  
The n-p initial state in GENIE is weighted up by a factor of four
while the n-n initial state is weight down by a factor of four.
The linear reduction of the cross section for 2p2h-MEC events is
corrected producing a flat cross section above 1~GeV. The effect of
these two adjustements on the energy distributions is shown by the
dotted blue histogram (labelled ``fixInitState\&XSec'') in the Figures.
The dotted green curve (labelled ``Tufts2p2hWgt'') shows the MC
distributions after the 2p2h cross section is reweighted as a function
of $q_0$ and $\vec{q}$. 
This is done using comparisons between the stock GENIE MEC events and
the quasielastic events.
Finally, the dotted magenta histogram (labelled ``TuftsWeight'') shows
the MC distributions after all the above adjustments are made in
combination with eweighting GENIE deep inelastic scattering with an
invariant mass $< 1.7~\text{GeV}$ down by 35\%.~\cite{tuftsWeightNote}

As mentioned above, the dotted magenta histogram in
Figures~\ref{fig:tuftsWeight_hadE}, \ref{fig:tuftsWeight_hadERatio},
\ref{fig:tuftsWeight_muE}, \ref{fig:tuftsWeight_muERatio},
\ref{fig:tuftsWeight_nuE} and \ref{fig:tuftsWeight_nuERatio} shows the
final MC distributions after the adjustments described above are
applied. 
Figure~\ref{fig:tuftsWeight_hadERatio} shows that the data/MC
ratio for the hadronic energy distribution is improved after the
inclusion of 2p2h events along with the combination of
adjustments. However, there is still a large discrepency between the
near detector data and MC in the range $1 - 2.5~\text{GeV}$. This
means that the hadronic component of events will be worse resolved for
higher values of hadronic energy. 
Figure~\ref{fig:tuftsWeight_muERatio} shows the data/MC ratio for muon
energy. Before including MEC (red histogram) MC generally
underestimates the event counts for each muon energy bin, particularly
in the range $2-3~\text{GeV}$. After the including the 2p2h events and
making the adjustments (dotted magenta histogram) the MC
underestimates the event count below about 1~GeV, close to matches the
data between 1 and $1.5~\text{GeV}$, and overestimates the event count
per bin above $1.5~\text{GeV}$.
Figure~\ref{fig:tuftsWeight_nuERatio} shows the data/MC ratio for
neutrino energy. Compared to the simulation excluding 2p2h (red
histogram), the addition of 2p2h along with the adjustments 
improves the data/MC ratio below $1.75~\text{GeV}$ and in the range
$3.5 - 4~\text{GeV}$. However, the ratio worsens the in the range
$1.75 - 3.5~\text{GeV}$.


% - Use weighting function wMEC (eq. 1) to correct the basic MEC cross
% section behavior. $w_{MEC} = w_{NN} w_{XS}(E)$
% Weights up n-p initial state and weights down n-n initial state.
% Eliminates the linear die-off of the cross section.
%  = kDytmanMEC_FixItlState * kDytmanMEC_FixXsecEdep

% - Use weighting function wTufts (eq. 3) to reweight the MEC cross
% section as a function of MEC q0 and ~q to match the Tufts best
% prediction.  = kTufts2p2hWgt
% Reweight GENIE DIS with W < 1.7 GeV down by 35\%.
% = Added in last step (TuftsWeight, kFixNonres1Pi) in conjunction with above two


\begin{figure}
\centering
   \begin{subfigure}[b]{0.9\textwidth}
\includegraphics[width=1\linewidth]{../../../Analysis/thesisPlots/tuftsWeight/nd_data_mc_tuftsBrkDwn_Ehad.pdf}
   \caption{Comparison of hadronic energy distributions.}
   \label{fig:tuftsWeight_hadE} 
\end{subfigure}
\begin{subfigure}[b]{0.9\textwidth}
\includegraphics[width=1\linewidth]{../../../Analysis/thesisPlots/tuftsWeight/nd_data_mc_tuftsBrkDwn_RatioEhad.pdf}
   \caption{Comparison of data/MC for each bin of hadronic energy for
     each 2p2h-MEC weight applied to the MC.}
   \label{fig:tuftsWeight_hadERatio}
\end{subfigure}
\caption[]{Comparison between data and MC hadronic energy
  distributions in the near detector. MC distributions are shown for
  each stage of the modifications made to the additional 2p2h-MEC
  process included in the GENIE for the second analysis. The case
  where 2p2h events are not included is shown by the red histogram. }
\end{figure}

\begin{figure}
\centering
   \begin{subfigure}[b]{0.9\textwidth}
\includegraphics[width=1\linewidth]{../../../Analysis/thesisPlots/tuftsWeight/nd_data_mc_tuftsBrkDwn_Emu.pdf}
   \caption{Comparison of muon energy distributions.}
   \label{fig:tuftsWeight_muE} 
\end{subfigure}
\begin{subfigure}[b]{0.9\textwidth}
\includegraphics[width=1\linewidth]{../../../Analysis/thesisPlots/tuftsWeight/nd_data_mc_tuftsBrkDwn_RatioEmu.pdf}
   \caption{Comparison of data/MC for each bin of muon energy for
     each 2p2h-MEC weight applied to the MC.}
   \label{fig:tuftsWeight_muERatio}
\end{subfigure}
\caption[]{Comparison between data and MC muon energy
  distributions in the near detector. MC distributions are shown for
  each stage of the modifications made to the additional 2p2h-MEC
  process included in the GENIE for the second analysis. The case
  where 2p2h events are not included is shown by the red histogram. }
\end{figure}


\begin{figure}
\centering
   \begin{subfigure}[b]{0.9\textwidth}
\includegraphics[width=1\linewidth]{../../../Analysis/thesisPlots/tuftsWeight/nd_data_mc_tuftsBrkDwn_Enu.pdf}
   \caption{Comparison of neutrino energy distributions.}
   \label{fig:tuftsWeight_nuE} 
\end{subfigure}
\begin{subfigure}[b]{0.9\textwidth}
\includegraphics[width=1\linewidth]{../../../Analysis/thesisPlots/tuftsWeight/nd_data_mc_tuftsBrkDwn_RatioEnu.pdf}
   \caption{Comparison of data/MC for each bin of neutrino energy for
     each 2p2h-MEC weight applied to the MC.}
   \label{fig:tuftsWeight_nuERatio}
\end{subfigure}
\caption[]{Comparison between data and MC neutrino energy
  distributions in the near detector. MC distributions are shown for
  each stage of the modifications made to the additional 2p2h-MEC
  process included in the GENIE for the second analysis. The case
  where 2p2h events are not included is shown by the red histogram. }
\end{figure}

\section{Evaluation of Systematics}\label{sec:systs}

For the second muon neutrino disappearance analysis the effect of
systematic uncertainties on the neutrino, muon and hadronic energy
distributions were studied and quantified in terms of shifts to the
mean and integral of the energy distributions after applying the
appropriate systematic shift to the simulation.
Monte Carlo samples were created with systematic shifts or alternative
physics models (such as noise modelling) applied before the
reconstruction stage of the simulation.



\subsection{Calibration}\label{sec:calibsysts}
xy systematic and y shift systematic

The data/MC ratio of calibrated muon energy as a function of distance
from the cell centers are used to define calibration shape
systematics for the x and y-view cells independentantly. 
The y-view cells display the most extreme disagrement between data and
MC in terms of calibrated response along the cells. In comparison the
disagreement between data and MC in the x-view cells is negligable and
is currently neglected in the analysis of systematics\cite{dianaAbsCal}.

A comparison of proton energy in the data and MC is
used to define an uncertainty of $\pm 5$  for the energy scale of
hadronic showers\cite{SusanProtondEdxCalib}. The effect of the 5\%
calibration uncertainty is evaluted using MC samples generated with a
$+$5\% or $-$5\% shift applied.



\subsection{Birks}\label{sec:birkssysts}
Birks B,C, ?
Expand on rock muon issue 



\subsection{Noise}\label{sec:noisesysts}



