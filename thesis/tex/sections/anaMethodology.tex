
%Introduce the following chapter


\section{Reconstruction}\label{sec:reco}

The NOvA analysis begins with a collection of hits from cells with APD 
signals above threshold. 

The reconstruction begins with a collection of above threshold APD
signals clustered in space and time \cite{baird2015analysis,
  ester1996density}. 
The clusters of APD signals are used to reconstruct event
candidates.
The trajectories of charged particles within an event are reconstructed
by applying a technique based on the Kalman filter
algorythm~\cite{kalman1960new}. 



\section{Selection and Background}\label{sec:sel}

A series of selection packages are used to identify muon neutrino
events within the detector and also to reject background events.  
There two main sources of background to the muon neutrino
disappearance analysis are from cosmic ray muons and beam induced
backgrounds including neutral current events, $\nu_e$ charged current
and $\nu_\tau$ charged current events. 

The beam background events are estimated using the simulation for each
detector.
Events passing the selection but failing a truth cut
requiring a muon or anti-muon neutrino are deemed to be
background. 
% The beam background is predominantly due to tau, electron 
% and neutral-current neutrino interactions within the detector.

The cosmic background (mostly secondary cosmic ray muons) is estimated
using two samples of far detector data which occur outside of the beam
spill window.  
The first sample is taken using the timing sidebands of the data
collected with the NuMI trigger. This sample matches the exposure of the
detector to the NuMI beam but the shape of the neutrino energy
distribution is statistically limited.
The second sample is taken using a cosmic trigger where data is
collected for activity within the detector outside of the beamspill window.

Figure~\ref{fig:cosCutFlow} shows the estimated number of cosmic
background and signal events after each successive selection is
applied to the sample.
As shown in Figure~\ref{fig:cosCutFlow}, selections are made for good
spills, data qualtiy, cosmic rejection, containment and neutral
current rejection. 

The first selection, at the top of Figure~\ref{fig:cosCutFlow}, is the
good spill selection which requires that the
NuMI beam is produced within accepatble bounds of spill time ($< 0.5
\times 10^9~\text{seconds}$), 
spill POT ($> 2 \times 10^{12}$), 
horn current ($-202~\text{kA} < I < -198~\text{kA}$), proton beam
position on NuMI target ($-2~\text{mm} < pos(x,y) < 2~\text{mm}$) and
beam width ($0.57~\text{mm} < width(x,y) <
1.58~\text{mm}$)\cite{NOvASADAQ}~\cite{NOvAFABeamMon}.  

Next, the data quality selection removes events with problems in one
or more data concentrator modules. 
The selection requires that: no data concentrator modules completely
drop out during the spill, 


%good spills

%explain data quality

% explain the cosmic BDT

%explain containment



% include plot of events after each selection level
\begin{figure}
  \centering
  \includegraphics[width=0.9\textwidth]{../../img/selection/cutflow.pdf}
  \caption{The number of signal events (red) and cosmic background
    events (yellow) surviving each successive analysis selection. The
    signal is estimated from the simulation and the cosmic background
    is estimated from the timing sidebands of the NuMI trigger. Figure
    taken from~\cite{NOvANumuSABless}.
  } 
  \label{fig:cosCutFlow}
\end{figure}

% explain the ReMId selection and the CVN selection
A k-nearest neighbours (kNN) classifier~\cite{altman1992introduction}
known as Reconstructed Muon Identification
is used to identify muon candidates among the particle
trajectories within an event~\cite{raddatzThesis}. 
The Reconstructed Muon Identification algorythm uses the following
four variables as inputs to the kNN classifier: 
dE/dx, scattering, track length and fraction of plains along the track
consistent with additional hadronic energy depositions~\cite{raddatzThesis}.



\section{Calibration}\label{sec:calibration}


threshold, attentuation, energy scale, timing, drift


\section{Extrapolation}\label{sec:extrap}
The NOvA near detector is used to compare distributions resulting from
neutrino interactions in data and simulation. Any significant
difference between the two means that a process exists that is not
correctly modelled in the simulation. 

The differences in the near detector neutrino energy spectrum between
data and simulation are extrapolated to predict the far detector
neutrino energy spectrum. The extrapolation accounts for neutrino
oscillations, acceptance differences and flux differences between the
near and far detector. 

The extrapolation proceeds in stages. 
First, simulation estimated background is subtracted from the the near
detector data spectrum.  
A reconstructed to true neutrino energy matrix obtained from the near
detector simulation is used to convert the background subtracted
reconstructed neutrino energy into a true energy spectrum. 
A far-to-near detector event ratio is used to account for the effect
of neutrino oscillations and the different acceptances of the two
detectors.
The true neutrino energy spectrum is multiplied by the far to near
event ratio as a function of true neutrino energy to produce the far
detector true neutrino energy spectrum.  The far detector
true neutrino energy is converted back to reconstructed
neutrino energy using the far detector reconstructed to true neutrino
energy matrix obtained from the far detector simulation. 
Finally, background events due to cosmic rays (from data) and beam
backgrounds (from MC) are added to the extrapolated far detector
reconstructed neutrino energy distribution to form a prediction that
will later be compared to the far detector data.
\cite{NOvASA}



% Neutrino oscillation is a function, among other things, of true
% neutrino energy. Therefore to predict the neutrino energy distribution
% in the FD after oscillations requires the truth information provided
% by the MC simulation.


\section{GENIE Tune}\label{sec:genietune}

Go through work done on the hadE task force. Show the plots of
hadronic energy comparing data and MC whilst applying the MEC weights
sequentially. 


\section{Evaluation of Systematics}\label{sec:systs}
show effect of each systematic and shift in Norm, mean energy and rms of neutrino energy.

\subsection{Calibration}\label{sec:calibsysts}
xy systematic and y shift systematic

\subsection{Birks}\label{sec:birkssysts}
Birks B,C, ?
Expand on rock muon issue 

\subsection{Noise}\label{sec:noisesysts}
