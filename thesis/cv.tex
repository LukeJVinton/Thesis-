\documentclass[10pt]{article} % Default font size
\pagenumbering{gobble}


\begin{document}


% \chapter{Curriculum Vitae}
% \addcontentsline{toc}{chapter}{Curriculum Vitae}

{\bf \noindent Luke Vinton } \\
University of Sussex\\
Department of Physics and Astronomy \\
Pevensey 2 Building, Room 4A23 \\
Brighton, BN1 9QH \\
Phone:~(+44)7857837124 \\
%
%=======================================================================
% Education
%
\medskip \\
{\bf EDUCATION ~~\hrulefill}
\medskip \\
%=======================================================================
\makebox[2.2in][l]{University of Sussex}
\makebox[2.5in][l]{{\bf Ph.D.}, High Energy Physics}
\makebox[1.0in][l]{2013-2017} \\%=======================================================================
\makebox[2.2in][l]{Unversity of Manchester}
\makebox[2.5in][l]{{\bf MPhys}, Physics with Astrophysics}
\makebox[1.0in][l]{2009-2013} \\%=======================================================================
%\makebox[2.2in][l]{T.I.S. Bookstore}
%\makebox[2.5in][l]{{\bf Shipping Clerk}}
%\makebox[1.0in][l]{2001-2005} \\
%=======================================================================
%
% Research
%
\medskip \\
{\bf RESEARCH: The NOvA Experiment, 2013-2017~~\hrulefill}
\medskip \\
\makebox[5.5in][l]{\underline{\it Analysis:}}
% $\bullet$ The NOvA experiment is a long-baseline, beam neutrino oscillation experiment based out of Fermilab.  It uses the recently upgraded NuMI beam line to search for the appearance of electron neutrinos and the disappearance of muon neutrinos in a predominantly muon neutrino beam.  It makes use of two detectors, a 300 ton near detector located underground at Fermilab and a 14 kiloton far detector located 810 km away in Ash River, MN.  Both detectors use a liquid scintillator with polystyrene fibers and avalanche photo diodes for particle detection.
\\
%
$\bullet$ {\bf Thesis:} {\it A Study of Muon Neutrino Disappearance
  With The NOvA Detectors And The NuMI Beam },
Adviser: Dr. Jeff Hartnell. Graduation date: March, 2017
\\
%
$\bullet$ Developed methods to seperate muon neutrino events by energy resolution
\\

% %
% $\bullet$ Enhanced a modified Hough transform to identify major event features used as a first stage in a global vertex identification algorithm.
% \\

%
\medskip \\
\makebox[5.5in][l]{\underline{\it Detector Monitoring and Commissioning:}}
\\
$\bullet$ Developed a series of online monitoring tools that provide real-time feedback on critical detector performance. These tools display histograms of key variables and provide the ability to generate plots comparing current and previous detector states.
\\
%
$\bullet$ Created a fully automated series of scripts for nearline processing that produce continually updated metrics posted to a webpage monitored by shifters. This information is also used for data quality purposes to determine good runs and provide bad channel masks for both of the NOvA detectors.
\\
%
$\bullet$ Installed and maintained software across eight different computers necessary for running the online and nearline monitoring software. This included being the 24-hour on-call expert for all of the data-monitoring software running on these eight machines for over a year.
\\
%
$\bullet$ Created the software to map a coordinate system based on electronic hardware elements to physical locations in the NOvA detectors for the data acquisition group.
\\
%
$\bullet$ Spent one month assisting in the installation of hardware and cabling for the NOvA near detector and prototype detector.
\\
%
%\newpage
\medskip \\
{\bf ADDITIONAL RESEARCH EXPERIENCE~~\hrulefill}
\medskip \\
\makebox[4.5in][l]{\underline{\it Long-Baseline Neutrino Facility: 2011-2012}}
%\makebox[1.0in][r]{2011-2012}
% $\bullet$ The Long Baseline Neutrino Experiment (LBNE) is a proposed neutrino experiment based out of Fermilab and making use of the existing NuMI beam line.  It will use liquid argon as a scintillator within a time projection chamber for particle detection.
\\
%
$\bullet$ Assisted with the research and development of plastic bars coated with a wavelength shifting material to be used to capture scintillation light within the LBNF detectors.
\\
%
\makebox[4.5in][l]{\underline{\it Research Assistant, Indiana University Dept. of Physics: 2010}}
%\makebox[1.0in][r]{2010}
\\
%
$\bullet$ Created a numerical simulation of a low energy deuteron beam used to model a pyro-electric fusion device.
\\
%
\makebox[4.5in][l]{\underline{\it Research Assistant, Indiana University Dept. of Astronomy: 2000-2001}}
%\makebox[1.0in][r]{2000-2001}
\\
%
$\bullet$ Performed computational simulations of gravitationally interacting few-body systems. Independently developed software to analyze the statistics of the results from thousands of different simulations.
\\
%
\makebox[4.5in][l]{\underline{\it Research Assistant, Harvard-Smithsonian Center for Astrophysics: 1998}}
%\makebox[1.0in][r]{1998}
\\
%
$\bullet$ Worked with Dr. Peter Garnavich and Dr. Eric Schlegel to analyze the type IIn supernova 1995N by processing and analyzing 25 to 30 photometric and spectroscopic digital pictures.
\\
%
%== End Research block
%
%=======================================================================
%
% Presentations
%
\\
\\
\\
\\
\medskip \\
{\bf PRESENTATIONS~~\hrulefill }
\medskip \\
$\bullet$ {\bf `Event Reconstruction with the NOvA Far Detector':}
%\begin{adjustwidth}{2.5em}{0pt}
  Poster presented at the Physics in Collision 2014 conference, September 2014, Bloomington, IN \\
%\end{adjustwidth}
%
$\bullet$ {\bf `Event Reconstruction with the NOvA Far Detector':}
%\begin{adjustwidth}{2.5em}{0pt}
  Poster presented at the Neutrino 2014 conference, June 2014, Boston, MA \\
%\end{adjustwidth}
%
$\bullet$ {\bf `Expected Sensitivities from the $\nu_{\mu}$ Disappearance Analysis Using the NOvA Detector':}
%\begin{adjustwidth}{2.5em}{0pt}
  Talk given at the American Physical Society Division of Particles and Fields 2013 conference, August 2013, Santa Cruz, CA \\
%\end{adjustwidth}
%
$\bullet$ {\bf `An Analysis of the Peculiar Type IIn Supernova 1995N':}
%\begin{adjustwidth}{2.5em}{0pt}
  Poster presented at the 193$^{rd}$ meeting of the American Astronomical Society, January 1999, Austin, TX
%\end{adjustwidth}
%
% link to abstract for this poster: http://aas.org/archives/BAAS/v30n4/aas193/461.htm
%
%== End presentation block

\end{document}